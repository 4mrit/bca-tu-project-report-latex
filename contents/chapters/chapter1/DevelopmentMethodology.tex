\section{Development Methodology}

This project follows the Iterative and Incremental Development Methodology, which allows the system to be built in small parts, tested frequently, and improved at each cycle. This method is suitable because the project grows feature-by-feature, and feedback can be applied immediately.

\begin{figure}[h]
  \centering
  \includegraphics[width=1\linewidth]{contents/chapters/chapter1/images/Iterative-and-incremental-agile-development-process-source-agile-development-toolscom.jpg}
  \caption{Iterative and Incremental Development Model}\label{fig:example}
\end{figure}

\newpage
%\textbf{Steps of the Methodology:}
\noindent\subsection*{Steps of the Methodology}

\renewcommand{\labelenumi}{\roman{enumi}.}
\begin{enumerate}[leftmargin=*]
    \item \textbf{Requirement Study} \\
    The project started by identifying the core needs of a small e-commerce platform. 
    This included understanding how products, carts, orders, and admin management should work.
    
    \item \textbf{Planning and Design} \\
    After gathering requirements, the system structure was planned. Database tables were designed, 
    user interface layouts were sketched, and the main modules were defined.
    
    \item \textbf{Development (Incremental Implementation)} \\
    Features were developed in small increments rather than all at once. 
    Each increment added a working feature such as product listing, cart, checkout, or admin tools.
    
    \item \textbf{Testing} \\
    Every new feature was tested immediately after development. 
    Manual testing was used to check functionality, usability, and any errors.
    
    \item \textbf{Review and Improvement} \\
    Based on testing results, necessary fixes and improvements were made. 
    After each cycle, the system became more stable and closer to the final version.
\end{enumerate}
