\subsection{Test Cases For Unit Testing}

\begin{table}[h]
    \setstretch{1.1}
    \centering
    \noindent
    \begin{tabularx}{\textwidth} { 
        | >{\centering\arraybackslash}X
        | >{\centering\arraybackslash}X
        | >{\centering\arraybackslash}X
        | >{\centering\arraybackslash}X
        | >{\centering\arraybackslash}X | }
        \hline
        \textbf{Test Case No} & \textbf{Module} & \textbf{Test Description} & \textbf{Expected Result} & \textbf{Result} \\
        \hline
        TC-UT-001 & Authentication & JWT token generation with valid user credentials & Valid JWT token returned with user claims & Pass \\
        \hline
        TC-UT-002 & Authentication & JWT token validation with expired token & Token validation fails, returns unauthorized & Pass \\
        \hline
        TC-UT-003 & Product Catalog & Product creation with valid data & Product saved to database, ID generated & Pass \\
        \hline
        TC-UT-004 & Product Catalog & Product price validation (negative/zero prices) & Validation fails, error message returned & Pass \\
        \hline
        TC-UT-005 & Order Processing & Order total calculation with multiple items & Correct total including tax and shipping calculated & Pass \\
        \hline
        TC-UT-006 & Order Processing & Inventory update on order placement & Product stock reduced by ordered quantity & Pass \\
        \hline
    \end{tabularx}
    \newline
    \caption{Unit Testing Test Cases}
\end{table}

\vfill
